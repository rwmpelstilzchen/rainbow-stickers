\documentclass{article}
%\usepackage[margin=0.5cm]{geometry}
\usepackage{geometry}
\pagestyle{empty}

\usepackage{fontspec}
\setmainfont{Alef}

\usepackage{xcolor, graphicx}
\usepackage{tikz}
\pagecolor{black}

\usepackage{expl3}
\ExplSyntaxOn
\cs_new_eq:NN \Repeat \prg_replicate:nn
\ExplSyntaxOff

\tolerance=99999

\usepackage{textcase}

\usepackage{setspace}

\newcommand{\ar}[1]{\RL{\fontspec{AlMothnna}#1}}
\newcommand{\en}[1]{\LR{\fontspec{Alef}\textsc{#1}}}
\newcommand{\cop}[1]{\LR{\fontspec{Antinoou}#1}}
\newcommand{\el}[1]{\LR{\fontspec{GFS Didot}#1}}
\newcommand{\he}[1]{\RL{\fontspec[Script=Hebrew]{Alef}#1}}
\newcommand{\bibhe}[1]{\RL{\fontspec[Script=Hebrew]{SBL Hebrew}#1}}
\newcommand{\url}[1]{\LR{\fontspec{PragmataPro}#1}}

\newcommand{\sticker}[4]{\begin{minipage}{#1}\centering\color{magenta!75!white}#2\\[0.75ex]\includegraphics[width=#1]{#4}\\#3\end{minipage}}
	\newcommand{\stickerline}[5]{\Repeat{#2-1}{\sticker{#1}{#3}{#4}{#5}\hspace{0.45cm}}\sticker{#1}{#3}{#4}{#5}\vspace{0.5cm}
	
	}
	\newcommand{\stickerpage}[6]{\Repeat{#3}{\stickerline{#1}{#2}{#4}{#5}{#6}}\newpage}

\newcommand{\hl}[1]{\textcolor{magenta}{\textbf{#1}}}

\usepackage{tabulary}
\newcommand{\site}[3]{\hl{#2} & #3 & \small\url{#1}\\}
\newcommand{\iconexplanation}[3]{{\small\en{#3}} & #2 & \includegraphics[width=1cm]{flag_#1.eps}\\}

\newcommand{\symbolglyph}[1]{{\fontspec{Symbola}#1}}

\usepackage{needspace}
\newcommand{\flagline}{{\vspace{-1ex}~\hfill\includegraphics[angle=90, width=0.5\textwidth, height=0.05cm]{flag.eps}\hfill~}\needspace{4\baselineskip}}

\usepackage{datetime}
\yyyymmdddate
\renewcommand{\dateseparator}{\mbox{$ $}·\mbox{$ $}}

\usepackage{bidi}



\begin{document}

\newgeometry{margin=2.0cm}
\color{magenta}
\setRL\fontspec[Script=Hebrew]{Alef}
\resizebox{\linewidth}{!}{\hl{סטיקרים למען שחרור מגדרי ומיני!}}
\includegraphics[width=\textwidth, height=1ex]{flag.eps}

\vspace{0.5cm}

{
%\centering
%\large
\setlength{\parindent}{0cm}
\setlength{\parskip}{0.5cm}
\setstretch{1.1}

היי,

אנחנו חיות בחברה \hl{הטרונורמטיבית פטריארכלית}. מהאולטרסאונד ועד המוות, מכל כיוון נשפכים עלינו נהרות של מסרים, הנחות־יסוד, הכְוונות, ולכל כיוון שנסתכל נראה הפרדה, היררכיה, דיכוי.

אחרי יותר מ\hl{מאה שנים} של פמיניזם מודרני ו\hl{כמה וכמה עשורים} של פעילות להטב״קית, המאבק לשחרור מגדרי ומיני, לשוויון מלא (ולא מתוך התנרמלות מתרפסת) ולשינוי שורשי של פני החברה עדיין לא הושלם, ולא במפתיע: הפטריארכיה ההטרונורמטיבית נעוצה עמוק בלב־לבה של החברה שלנו, ו\hl{לוקח זמן} והרבה מאמץ משותף כדי למוסס ולהסיר אותה משם. אל יאוש! מספיק להסתכל קצת אחורה בזמן ולראות במבט־על שהכיוון הוא חיובי.

אז \hl{מה הסטיקרים האלה} שבדפים הבאים? הם לא הולכים לשנות את העולם, אבל כן הולכים לתת עוד כמה דחיפות קטנות בכיוון, ברוחב 2.54 ס״מ כ״א (\en{Angry Inch}…). אל מול אותם נהרות מהפסקה הראשונה, אלה אולי אבנים קטנות שבונות סכר. אל מול המחיקה והשיח ההגמוני, קשת צבעונית של נִרְאוּת.

\flagline

\newcommand{\descitem}[2]{\item[\symbolglyph{◃}~#1\hspace{1em}]#2}
הבחירה ב\hl{סטיקרים קטנטנים} היא מכמה סיבות:
\begin{description}
	\descitem{\hl{סביבתית}:}{פחות נייר, פחות דיו, פחות לכלוך.}
	\descitem{\hl{יחס עלות-תועלת}:}{הן מבחינת העלויות הכלכליות של ההדפסה והן מבחינת המאמץ והטרחה. הבחירה בדגל הגאווה נותנת נראות גראפית גם בגודל קטן, ומושכת תשומת־לב, במיוחד במרחב העירוני האפרורי.}
	\descitem{\hl{צורך בפעולה מצד הקוראת}:}{כמו שמחקרים בפסיכולוגיה מראים, השקעה ומאמץ, פעולה אקטיבית, גוררת מתן כובד־ראש. הסטיקר הקטן מחייב מי שמסתקרןת לקרוא מה כתוב בו (ניסיתי לעצב את הסטיקרים, מבחינת סקאלת הצבעים האנרכו־קווירית והקומפוזיציה, באופן כזה שימשכו את העין ויעוררו סקרנות) להתקרב ולקרוא. ברגע שגרמנו לה לעשות פעולה ולהתקרב, קיבלנו את הקשב שלה, בתקווה שתהיה לכך השפעה, בין אם בעידוד, בין עם בזריעה של רעיונות חדשים ובין אם בעימות עם תפיסות קיימות.}
	\descitem{\hl{מגוון}:}{כשמכינים סטיקרים קטנים הרבה יותר קל לגעת בנושאים מגוונים, כשכל אחד מהסטיקרים נוגע בעניין אחד או שניים. אם מדביקים הרבה סטיקרים שונים באותו האיזור, זה יכול לעורר „חדוות איסוף” נוסח פוגים או פוקימון… \symbolglyph{☺}, ולהפוך אח המרחב הציבורי למשהו קצת יותר משחקי.

		אין יצוג שווה לכולן, לא בגלל היררכיזציה או הדרה חו״ח, אלא בגלל שעסקתי יותר בדברים שיש להם נגיעה אישית לי. אם את רוצה להוסיף עוד סטיקרים, נהדר, דברי איתי.}
	\descitem{\hl{גמישות}:}{סטיקרים קטנים אפשר להדביק בכל מני מקומות שמודעות גדולות יותר, או אפילו סטיקרים בגודל מלא, אי־אפשר, מה שמאפשר שימושים מקוריים.}
\end{description}
אני מודעת לכך שלא כולן יכולות לקרוא \hl{אותיות קטנות} בקלות. גודל האותיות הוא פחות או יותר זה שמשמש בספרים, ובחרתי בפונט קריא במיוחד גם בגדלים קטנים, הפונט החופשי „אלף” (\en{alef.hagilda.com}).

\flagline

\hl{רוצות להשתתף?} נהדר! כמה דרכים:
\begin{description}
	\descitem{\hl{להדפיס, לגזור ולהדביק בעצמכן}.}{כדאי להדפיס את הדפים שמעניינים אתכן על נייר דביק (אמור להיות בכל בית־דפוס או מרכז לשירותי הדפסה, ולא בעיה להשיג גם למדפסת ביתית), אבל אפשר גם עם סלוטייפ. בבקשה תתחשבו באחרות בכדור־הארץ ואל תציפו באופן פולשני או לא נעים בדרך אחרת: זה גם קאונטר-אפקטיבי (איך אומרים את זה בעברית?) וגם סתם התנהגות מגעילה. בחירת מקומות חכמה עדיפה בהרבה על סתם הדבקה של מלא סטיקרים. שימי לב: צבעים רגילים של מדפסות דוהים בשמש; כדאי לבחור אולי מקומות שלא חשופים לשמש ישירה. כנ״ל לגבי גשם.}
	\descitem{\hl{לקבל ממני סטיקרים}.}{לא רוצה להדפיס? אשמח להעביר אליך כמות. לקשירת קשר: \en{foo@digitalwords.net} / 02-6419913.}
	\descitem{\hl{רעיונות חדשים}.}{יש לך רעיון מגניב לסטיקרים נוספים? נפלא, צרי קשר. אפשר גם להוריד את הקוד, ב־\en{\XeLaTeX} ו־\en{EPS}, מהכתובת \en{digitalwords.net/?p=905}, ולהכין לבד. אשמח גם לקבל תוספות מעניינות לרשימת הבלוגים שלמטה.}
	\descitem{\hl{לתרגם}.}{לא את הכל אפשר לתרגם בקלות, בגלל משחקי־מילים, חריזה, אידיומטיות, מילים וביטויים בלתי־תרגימים ושאר דברים תלויי־שפה, אבל זה יהיה ממש נהדר אם יהיו תרגומים~— במיוחד לערבית ורוסית, אבל גם יידיש, מאלאיאלאם וקוריאנית הולך…}
\end{description}

\flagline

אם אהבתן את הסטיקרים כאן, די סביר להניח שהחוברת הקטנה והמגניבה „\hl{איך להפסיק להיות סטרייט/ית ב־7 צעדים קלים}” (\en{tiny.cc/str8it}) תדבר אליכן. יכול להיות שיעניינו אתכן גם ה\hl{בלוגים} הבאים:

\begin{tabulary}{\textwidth}{RCL}
	\site{bligeula.wordpress.com}{בלי גאולה}{יוסףה מקייטון}
	\site{the-big-sister.com}{האחות הגדולה}{}
	\site{israel.ihollaback.org}{הכצעקתה}{}
	\site{ponetium.wordpress.com}{הלא אנושית מהר כרוב}{פונטיום חציעץ}
	\site{parshedona.wordpress.com}{הפרשדונה}{דן וג׳}
	\site{meandiscourse.wordpress.com}{יחסי מין}{}
	\site{vistabul.wordpress.com}{מפיסקס פמיניסטי}{שרון אורשלימי}
	\site{queereyeworld.wordpress.com}{ע׳ קווירית}{עינב בר}
	\site{uprootingroots.wordpress.com}{עיקור שורשים}{}
	\site{reuma0.wordpress.com}{ראומה}{}
	\site{profitsbeforelives.wordpress.com}{רווחים לפני חיים}{לילך בן־דוד}
	\site{bidyke.wordpress.com}{שחור-סגול}{שירי אייזנר}
	\site{nehashim.wordpress.com}{שק של נחשים}{דורה קישינבסקי}
	\site{radicalbi.wordpress.com bidyke.tumblr.com}{\en{Bi radical}}{שירי אייזנר}
	\site{feministfrequency.com}{\en{Feminist Frequency}}{אניטה סרקיזיאן}
	\site{lilithlaughed.co.uk}{\en{Lilith Laughed}}{טלי יאנר־קלאוזנר}
	%\site{radicalbi.wordpress.com}{\en{Radical bi}}{שירי אייזנר באנגלית}
	\hspace{0.33\linewidth} &
	\hspace{0.33\linewidth} &
	\hspace{0.33\linewidth}
\end{tabulary}

\vfill

\begin{center}
	\small זו \hl{גרסה} של הקובץ מתאריך \hl{\LR{\today}}
	
	הגרסה \hl{האחרונה} זמינה בכתובות הבאות:\\
	\en{digitalwords.net/?p=905}\\
	\en{https://github.com/rwmpelstilzchen/rainbow-stickers}
\end{center}

%  \begin{tabulary}{\textwidth}{LCR}
%  	\iconexplanation{barbedwire}{הפרדה מגדרית}{***}
%  	\iconexplanation{curseword}{קללה}{***}
%  	\iconexplanation{cursor}{עצמיניות}{***}
%  	\iconexplanation{exclamation}{פומביות}{***}
%  	\iconexplanation{genderanarchy}{אנרכיה מגדרית}{***}
%  	\iconexplanation{genderqueer}{ג׳נדרקוויר}{***}
%  	\iconexplanation{heart}{אהבה}{***}
%  	\iconexplanation{intersex}{אינטרסקס}{***}
%  	\iconexplanation{knife}{חיתוך תינוקות}{***}
%  	\iconexplanation{megaphone}{פוליטיות}{***}
%  	\iconexplanation{pansexuality}{פאנסקסואליות}{***}
%  	\iconexplanation{pointing}{אקסופוריות\break\small (להדביק על מודעות וכד׳)}{***}
%  	\iconexplanation{polyamory}{פוליאמוריה}{***}
%  	\iconexplanation{questionmark}{שאלה}{***}
%  	\iconexplanation{smile}{אופטימיות}{***}
%  	\iconexplanation{toilet}{הפרדה מגדרית 2}{***}
%  %	\iconexplanation{}{}{***}
%  \end{tabulary}
}


%\en{http://commons.wikimedia.org/wiki/File:Speech_balloon.svg}
%\en{http://commons.wikimedia.org/wiki/File:Megaphone-Vector.svg}

\newgeometry{top=0.5cm, bottom=0cm, right=0cm, left=0cm}
\newpage
\centering


%%% barbedwire
\stickerpage{2.54cm}{7}{8}{\he{שִׁחְרוּר}}{\he{מִגְּדֵרוֹת הַמִּגְדָּר}}{flag_barbedwire.eps} 
\stickerpage{2.75cm}{6}{7}{\he{אין \hl{הפרדה} מגדרית\\בלי \hl{דיכוי} מגדרי}}{\he{אין \hl{דיכוי} מגדרי\\בלי \hl{הפרדה} מגדרית}}{flag_barbedwire.eps} 
\stickerpage{2.75cm}{6}{4}{\en{No gender \hl{segregation} without gender \hl{oppression}}}{\en{No gender \hl{oppression} without gender \hl{segregation}}}{flag_barbedwire.eps}

%%% bicycle
\stickerpage{2.54cm}{7}{8}{\he{אשה בלי גבר}}{\he{היא כמו דג\\בלי אופניים}}{flag_bicycle.eps} 
\stickerpage{2.54cm}{7}{6}{\en{A woman without a man}}{\en{is like a fish without\\a bicycle}}{flag_bicycle.eps} 

%%% children
\stickerpage{2.54cm}{7}{9}{\he{גדלו בנים ובנות}}{\he{באותו האופן}}{flag_children.eps} 
\stickerpage{2.54cm}{7}{7}{\en{Raise boys\\and girls}}{\en{the same way}}{flag_children.eps} 

%%% curseword
\stickerpage{2.54cm}{7}{9}{\he{הומו}}{\he{זה לא קללה}}{flag_curseword.eps} 
\stickerpage{2.54cm}{7}{7}{\he{קוקסינל}}{\he{זה „פרת משה רבינו” בצרפתית,\\לא קללה}}{flag_curseword.eps}

%%% cursor
\stickerpage{2.54cm}{7}{8}{\he{עַצְמִינִיּוּת}}{\he{לאונן זה\\נעים וטוב}}{flag_cursor.eps} 
\stickerpage{2.54cm}{7}{9}{\en{Masturbation}}{\en{Liberation}}{flag_cursor.eps} 
\stickerpage{2.54cm}{7}{8}{\he{עַצְמִינִיּוּת}}{\he{לאהוב את עצמך}}{flag_cursor.eps} 

%%% drag
\stickerpage{2.54cm}{7}{7}{\en{We're born naked}}{\en{and the rest\\is drag}}{flag_drag.eps} 
\stickerpage{2.54cm}{7}{8}{\he{כולנו נולדים עירומים}}{\he{השאר זה דראג}}{flag_drag.eps} 

%%% exclamation
\stickerpage{2.54cm}{7}{8}{\he{נו, אז\\שהשכנים יגידו…}}{~}{flag_exclamation.eps}

%%% genderanarchy
\stickerpage{2.54cm}{7}{7}{\en{Gender anarchy}}{\en{will smash patriarchy}}{flag_genderanarchy.eps} 
\stickerpage{2.54cm}{7}{7}{\he{במגדר עשי אנרכיה}}{\he{ומגרי ת׳פטריארכיה}}{flag_genderanarchy.eps} 
\stickerpage{2.54cm}{7}{9}{\he{אנרכיה}}{\he{מגדרית}}{flag_genderanarchy.eps} 
\stickerpage{2.54cm}{7}{8}{\he{מבלעין לנאות־סמדר}}{\he{נשחרר את המגדר!}}{flag_genderanarchy.eps}

%%% genderqueer 
\stickerpage{2.54cm}{7}{9}{\he{ג׳נדרקוויר}}{\he{זה אדיר!}}{flag_genderqueer.eps}

%%% hairy
\stickerpage{2.54cm}{7}{9}{\he{שיער}}{\he{זה נהדר!}}{flag_hairy.eps}

%%% heart
\stickerpage{2.54cm}{7}{8}{\bibhe{וְאָֽהַבְתָּ֥ לְרֵעֲךָ֖}}{\bibhe{כָּמ֑וֹךָ}}{flag_heart.eps}
%\stickerline{2.54cm}{5}{\el{καὶ ἀγαπήσεις τὸν πλησίον σου}}{\el{ὡς σεαυτόν}}{flag_heart.eps}
\stickerpage{2.54cm}{7}{8}{\cop{ⲟⲩⲟϩ ⲉⲕⲉⲙⲉⲛⲣⲉ}}{\cop{ⲡⲉⲕϣⲫⲏⲣ ⲙ̅ⲡⲉⲕⲣⲏϯ}}{flag_heart.eps}
\stickerpage{2.54cm}{7}{7}{\he{לאהוב את הגוף}}{\he{זה הרבה יותר\\כיף מלשנוא\\ולמשטר אותו}}{flag_heart.eps}

%%% intersex
\stickerpage{2.5cm}{7}{8}{\he{לא רק \en{XY} ו־\en{XX}}}{\he{די לדיכוי האינטרסקס!}}{flag_intersex.eps}

%%% knife
\stickerpage{2.54cm}{7}{7}{\he{בשביל „נורמליות”}}{\he{לא חותכים תינוקות}}{flag_knife.eps}
\stickerpage{2.54cm}{7}{7}{\en{You don't mutilate babies}}{\en{for the sake of ‘normality’}}{flag_knife.eps}
\stickerpage{2.54cm}{7}{6}{\he{חבל על כל \hl{מילה}}}{\he{בשביל „נורמליות”\\לא חותכים תינוקות}}{flag_knife.eps}

%%% megaphone
\stickerpage{2.54cm}{7}{9}{\he{האישי}}{\he{הוא פוליטי}}{flag_megaphone.eps} 
\stickerpage{2.54cm}{7}{9}{\en{The personal}}{\en{is political}}{flag_megaphone.eps}

%%% pansexuality
\stickerpage{2.54cm}{7}{7}{\he{לא רוצה להיות נורמלית}}{\he{גאווה פאנסקסואלית!}}{flag_pansexuality.eps} 
\stickerpage{2.54cm}{7}{7}{\he{לא רוצה להיות נורמלית}}{\he{גאווה ביסקסואלית!}}{flag_pansexuality.eps} 
\stickerpage{2.54cm}{7}{6}{\he{לֹא טוֹעָה\\לֹא מְבֻלְבָּל}}{\he{גֶ׳נְדֶרקְוִיר פַּנְסֶקְסוּאָל!}}{flag_pansexuality.eps}
\stickerpage{2.54cm}{7}{6}{\he{לֹא טוֹעָה\\לֹא מְבֻלְבָּל}}{\he{גֶ׳נְדֶרקְוִיר בִּיסֶקְסוּאָל!}}{flag_pansexuality.eps}

%%% pointing
\stickerpage{2.54cm}{7}{8}{\he{\LARGE ככה}}{\he{נראית הבניה מגדרית}}{flag_pointing.eps} 
\stickerpage{2.54cm}{7}{9}{\he{\LARGE ככה}}{\he{נראה שוביניזם}}{flag_pointing.eps} 
\stickerpage{2.54cm}{7}{9}{\he{\LARGE ככה}}{\he{נראית מיזוגניה}}{flag_pointing.eps} 
\stickerpage{2.54cm}{7}{8}{\he{\LARGE ככה}}{\he{נראית תעמולה הטרונורמטיבית}}{flag_pointing.eps} 

%%% polyamory 
\stickerpage{2.54cm}{7}{7}{\he{אהבה אינה\\יודעת מגדר}}{\he{היא גם לא\\יודעת לספור}}{flag_polyamory.eps} 
\stickerpage{2.54cm}{7}{9}{\he{פוליאמוריה}}{\he{אהבה לאין גבול}}{flag_polyamory.eps}
\stickerpage{2.54cm}{7}{8}{\he{פוליאמוריה}}{\he{\hl{לאהוב} זה \hl{הרבה} יותר טוב}}{flag_polyamory.eps}

%%% questionmark
\stickerpage{2.54cm}{7}{7}{\he{איך להפסיק להיות סטרייט/ית ב־7 צעדים קלים}}{\en{tiny.cc/str8it}}{flag_questionmark.eps}

%%% smile
\stickerpage{2.54cm}{7}{9}{\he{קוםי, התנערי,}}{\he{ושני את העולם}}{flag_smile.eps}
\stickerpage{2.54cm}{7}{8}{\en{Cheer up}}{\en{and change\\the world}}{flag_smile.eps}

%%% toilet 
\stickerpage{2.54cm}{7}{7}{\he{7,000,000,000 אנשים}}{\he{הרבה יותר\\משני מגדרים}}{flag_toilet.eps}
\stickerpage{2.54cm}{7}{7}{\he{כשאין שירותים להכנס אליהם}}{\he{מחרבנים על ההפרדה המגדרית}}{flag_toilet.eps}


\end{document}
