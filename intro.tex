\newgeometry{margin=2.0cm}
\color{magenta!75!white}
\setRL\fontspec[Script=Hebrew]{Alef}
\resizebox{\linewidth}{!}{\hl{סטיקרים למען שחרור מגדרי ומיני!}}
\includegraphics[width=\textwidth, height=1ex]{flag.eps}

\vspace{0.5cm}

{
%\centering
%\large
\setlength{\parindent}{0cm}
\setlength{\parskip}{0.5cm}
\setstretch{1.1}

היי,

אנחנו חיות בחברה \hl{הטרונורמטיבית פטריארכלית}. מהאולטרסאונד ועד המוות, מכל כיוון נשפכים עלינו נהרות של מסרים, הנחות־יסוד, הכְוונות, ולכל כיוון שנסתכל נראה הפרדה, היררכיה, דיכוי.

אחרי יותר מ\hl{מאה שנים} של פמיניזם מודרני ו\hl{כמה וכמה עשורים} של פעילות להטב״קית, המאבק לשחרור מגדרי ומיני, לשוויון מלא (ולא מתוך התנרמלות מתרפסת) ולשינוי שורשי של פני החברה עדיין לא הושלם, ולא במפתיע: הפטריארכיה ההטרונורמטיבית נעוצה עמוק בלב־לבה של החברה שלנו, ו\hl{לוקח זמן} והרבה מאמץ משותף כדי למוסס ולהסיר אותה משם. אל יאוש! מספיק להסתכל קצת אחורה בזמן ולראות במבט־על שהכיוון הוא חיובי.

אז \hl{מה הסטיקרים האלה} שבדפים הבאים? הם לא הולכים לשנות את העולם, אבל כן הולכים לתת עוד כמה דחיפות קטנות בכיוון, ברוחב 2.54 ס״מ כ״א (\en{Angry Inch}…). אל מול אותם נהרות מהפסקה הראשונה, אלה אולי אבנים קטנות שבונות סכר. אל מול המחיקה והשיח ההגמוני, קשת צבעונית של נִרְאוּת.

\flagline

\newcommand{\descitem}[2]{\item[\symbolglyph{◃}~#1\hspace{1em}]#2}
הבחירה ב\hl{סטיקרים קטנטנים} היא מכמה סיבות:
\begin{description}
	\descitem{\hl{סביבתית}:}{פחות נייר, פחות דיו, פחות לכלוך.}
	\descitem{\hl{יחס עלות-תועלת}:}{הן מבחינת העלויות הכלכליות של ההדפסה והן מבחינת המאמץ והטרחה. הבחירה בדגל הגאווה נותנת נראות גראפית גם בגודל קטן, ומושכת תשומת־לב, במיוחד במרחב העירוני האפרורי.}
	\descitem{\hl{צורך בפעולה מצד הקוראת}:}{כמו שמחקרים בפסיכולוגיה מראים, השקעה ומאמץ, פעולה אקטיבית, גוררת מתן כובד־ראש. הסטיקר הקטן מחייב מי שמסתקרןת לקרוא מה כתוב בו (ניסיתי לעצב את הסטיקרים, מבחינת סקאלת הצבעים האנרכו־קווירית והקומפוזיציה, באופן כזה שימשכו את העין ויעוררו סקרנות) להתקרב ולקרוא. ברגע שגרמנו לה לעשות פעולה ולהתקרב, קיבלנו את הקשב שלה, בתקווה שתהיה לכך השפעה, בין אם בעידוד, בין עם בזריעה של רעיונות חדשים ובין אם בעימות עם תפיסות קיימות.}
	\descitem{\hl{מגוון}:}{כשמכינים סטיקרים קטנים הרבה יותר קל לגעת בנושאים מגוונים, כשכל אחד מהסטיקרים נוגע בעניין אחד או שניים. אם מדביקים הרבה סטיקרים שונים באותו האיזור, זה יכול לעורר „חדוות איסוף” נוסח פוגים או פוקימון… \symbolglyph{☺}, ולהפוך אח המרחב הציבורי למשהו קצת יותר משחקי.

		אין יצוג שווה לכולן, לא בגלל היררכיזציה או הדרה חו״ח, אלא בגלל שעסקתי יותר בדברים שיש להם נגיעה אישית לי. אם את רוצה להוסיף עוד סטיקרים, נהדר, דברי איתי.}
	\descitem{\hl{גמישות}:}{סטיקרים קטנים אפשר להדביק בכל מני מקומות שמודעות גדולות יותר, או אפילו סטיקרים בגודל מלא, אי־אפשר, מה שמאפשר שימושים מקוריים.}
\end{description}
אני מודעת לכך שלא כולן יכולות לקרוא \hl{אותיות קטנות} בקלות. גודל האותיות הוא פחות או יותר זה שמשמש בספרים, ובחרתי בפונט קריא במיוחד גם בגדלים קטנים, הפונט החופשי „אלף” (\en{alef.hagilda.com}).

\flagline

\hl{רוצות להשתתף?} נהדר! כמה דרכים:
\begin{description}
	\descitem{\hl{להדפיס, לגזור ולהדביק בעצמכן}.}{כדאי להדפיס את הדפים שמעניינים אתכן על נייר דביק (אמור להיות בכל בית־דפוס או מרכז לשירותי הדפסה, ולא בעיה להשיג גם למדפסת ביתית), אבל אפשר גם עם סלוטייפ. בבקשה תתחשבו באחרות בכדור־הארץ ואל תציפו באופן פולשני או לא נעים בדרך אחרת: זה גם קאונטר-אפקטיבי (איך אומרים את זה בעברית?) וגם סתם התנהגות מגעילה. בחירת מקומות חכמה עדיפה בהרבה על סתם הדבקה של מלא סטיקרים. שימי לב: צבעים רגילים של מדפסות דוהים בשמש; כדאי לבחור אולי מקומות שלא חשופים לשמש ישירה. כנ״ל לגבי גשם.}
	\descitem{\hl{לקבל ממני סטיקרים}.}{לא רוצה להדפיס? אשמח להעביר אליך כמות. לקשירת קשר: \en{foo@digitalwords.net} / 02-6419913.}
	\descitem{\hl{רעיונות חדשים}.}{יש לך רעיון מגניב לסטיקרים נוספים? נפלא, צרי קשר. אפשר גם להוריד את הקוד, ב־\en{\XeLaTeX} ו־\en{EPS}, מהכתובת \en{digitalwords.net/?p=905}, ולהכין לבד. אשמח גם לקבל תוספות מעניינות לרשימת הבלוגים שלמטה.}
	\descitem{\hl{לתרגם}.}{לא את הכל אפשר לתרגם בקלות, בגלל משחקי־מילים, חריזה, אידיומטיות, מילים וביטויים בלתי־תרגימים ושאר דברים תלויי־שפה, אבל זה יהיה ממש נהדר אם יהיו תרגומים~— במיוחד לערבית ורוסית, אבל גם יידיש, מאלאיאלאם וקוריאנית הולך…}
\end{description}

\flagline

אם אהבתן את הסטיקרים כאן, די סביר להניח שהחוברת הקטנה והמגניבה „\hl{איך להפסיק להיות סטרייט/ית ב־7 צעדים קלים}” (\en{tiny.cc/str8it}) תדבר אליכן. יכול להיות שיעניינו אתכן גם ה\hl{בלוגים} הבאים:

\begin{tabulary}{\textwidth}{RCL}
	\site{bligeula.wordpress.com}{בלי גאולה}{יוסףה מקייטון}
	\site{the-big-sister.com}{האחות הגדולה}{}
	\site{israel.ihollaback.org}{הכצעקתה}{}
	\site{ponetium.wordpress.com}{הלא אנושית מהר כרוב}{פונטיום חציעץ}
	\site{parshedona.wordpress.com}{הפרשדונה}{דן וג׳}
	\site{meandiscourse.wordpress.com}{יחסי מין}{}
	\site{vistabul.wordpress.com}{מפיסקס פמיניסטי}{שרון אורשלימי}
	\site{queereyeworld.wordpress.com}{ע׳ קווירית}{עינב בר}
	\site{uprootingroots.wordpress.com}{עיקור שורשים}{}
	\site{reuma0.wordpress.com}{ראומה}{}
	\site{profitsbeforelives.wordpress.com}{רווחים לפני חיים}{לילך בן־דוד}
	\site{bidyke.wordpress.com}{שחור-סגול}{שירי אייזנר}
	\site{nehashim.wordpress.com}{שק של נחשים}{דורה קישינבסקי}
	\site{radicalbi.wordpress.com bidyke.tumblr.com}{\en{Bi radical}}{שירי אייזנר}
	\site{feministfrequency.com}{\en{Feminist Frequency}}{אניטה סרקיזיאן}
	\site{lilithlaughed.co.uk}{\en{Lilith Laughed}}{טלי יאנר־קלאוזנר}
	%\site{radicalbi.wordpress.com}{\en{Radical bi}}{שירי אייזנר באנגלית}
	\hspace{0.33\linewidth} &
	\hspace{0.33\linewidth} &
	\hspace{0.33\linewidth}
\end{tabulary}

\vfill

\begin{center}
	\small זו \hl{גרסה} של הקובץ מתאריך \hl{\LR{\today}}
	
	הגרסה \hl{האחרונה} זמינה בכתובות הבאות:\\
	\en{http://digitalwords.net/?p=905}\\
	\en{https://github.com/rwmpelstilzchen/rainbow-stickers}
\end{center}
}
